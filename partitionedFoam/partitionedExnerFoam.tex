%% LyX 2.2.3 created this file.  For more info, see http://www.lyx.org/.
%% Do not edit unless you really know what you are doing.
\documentclass[12pt,british]{article}
\usepackage{mathptmx}
\usepackage[T1]{fontenc}
\usepackage[latin9]{inputenc}
\usepackage[a4paper]{geometry}
\geometry{verbose,tmargin=2cm,bmargin=2cm,lmargin=2cm,rmargin=2cm}
\pagestyle{plain}
\setcounter{secnumdepth}{5}
\setcounter{tocdepth}{5}
\usepackage{array}
\usepackage{float}
\usepackage{bm}
\usepackage{amsmath}
\usepackage[authoryear]{natbib}

\makeatletter

%%%%%%%%%%%%%%%%%%%%%%%%%%%%%% LyX specific LaTeX commands.
%% Because html converters don't know tabularnewline
\providecommand{\tabularnewline}{\\}

%%%%%%%%%%%%%%%%%%%%%%%%%%%%%% User specified LaTeX commands.
\usepackage{color}
\newcommand{\nicefrac}[2]{\ensuremath ^{#1}\!\!/\!_{#2}}
\usepackage { fancybox}

\renewcommand{\floatpagefraction}{0.95}
\renewcommand{\textfraction}{0}
\renewcommand{\topfraction}{1}
\renewcommand{\bottomfraction}{1}

\makeatother

\usepackage{babel}
\begin{document}

\title{partionedExnerFoam \textendash{} numerical solution of the conditionally
averaged compressible Euler equations}

\author{Hilary Weller}
\maketitle

\section{Continuous Equations}

The compressible Euler equations can be conditionally averaged in
order to divide the atmosphere into partitions based on, for example,
the presence of buoyant updrafts, stable air and downdraughts. These
three partitions are labelled using $i$: 
\[
i=\begin{cases}
0 & \text{ stable}\\
1 & \text{ updraught}\\
2 & \text{ downdraught. }
\end{cases}
\]
Each partition has separate prognostic variables of density, $\rho_{i}$,
potential temperature, $\theta_{i}$, and velocity, $\mathbf{u}_{i}$,
but all regions share the same pressure, $p$ and Exner pressure,
$\pi=(p/p_{0})^{\kappa}$. Assuming that the atmosphere is a perfect
dry gas in a frame rotating with rate $\bm{\Omega}$, the conditionally
averaged Euler equations are:

\begin{eqnarray}
\frac{\partial\sigma_{i}\rho_{i}\mathbf{u}_{i}}{\partial t}+\nabla\cdot\left(\sigma_{i}\rho_{i}\mathbf{u}_{i}\mathbf{u}_{i}\right) & = & -2\sigma_{i}\rho_{i}\boldsymbol{\Omega}\times\mathbf{u}_{i}-\sigma_{i}\rho_{i}c_{p}\theta_{i}\nabla\pi+\sigma_{i}\rho_{i}\mathbf{g}\label{eq:condMom}\\
 & + & \sum_{j\ne i}\left(\sigma_{j}\rho_{j}\mathbf{u}_{j}S_{ji}-\sigma_{i}\rho_{i}\mathbf{u}_{i}S_{ij}-\sigma_{i}\sigma_{j}\mathbf{d}_{ij}\right)\nonumber \\
\frac{\partial\sigma_{i}\rho_{i}}{\partial t}+\nabla\cdot(\sigma_{i}\rho_{i}\mathbf{u}_{i}) & = & \sum_{j\ne i}\left(\sigma_{j}\rho_{j}S_{ji}-\sigma_{i}\rho_{i}S_{ij}\right)\label{eq:condCont}\\
\frac{\partial\sigma_{i}\rho_{i}\theta_{i}}{\partial t}+\nabla\cdot\left(\sigma_{i}\rho_{i}\mathbf{u}_{i}\theta_{i}\right) & = & \sum_{j\ne i}\left(\sigma_{j}\rho_{j}\theta_{j}S_{ji}-\sigma_{i}\rho_{i}\theta_{i}S_{ij}-\sigma_{i}\rho_{i}H_{ij}\right)\label{eq:condTheta}
\end{eqnarray}
with the global equation of state and the equation of state for each
partition:
\begin{eqnarray}
p_{0}\pi^{\frac{1-\kappa}{\kappa}} & = & R\rho\theta=R\sum_{i}\sigma_{i}\rho_{i}\theta_{i}\label{eq:condState}\\
p_{0}\pi^{\frac{1-\kappa}{\kappa}} & =R & \rho_{i}\theta_{i}\label{eq:condStatei}
\end{eqnarray}
and subject to
\begin{equation}
\sum_{i}\sigma_{i}=1.
\end{equation}
The prognostic variables are given in table \ref{tab:progs}, the
constants in table \ref{tab:constants} and the diagnostic variables
in table \ref{tab:diags}.

\begin{table}[H]
\begin{tabular}{|>{\raggedright}p{0.39\textwidth}|c|c|c|}
\hline 
 & Symbol & Value & Units\tabularnewline
\hline 
\hline 
Exner pressure & $\pi$ & $(p/p_{0})^{\kappa}$  & -\tabularnewline
\hline 
Potential temperature & $\theta_{i}$ & $T_{i}/\pi$  & K\tabularnewline
\hline 
Mass flux & $F_{i}$ & $\sigma_{i}\rho_{i}\mathbf{u}_{i}\cdot\mathbf{S}$ & $\text{kg s}^{-1}$\tabularnewline
\hline 
Mass in partition $i$ & $(\sigma_{i}\rho_{i})$ &  & $\text{kg m}^{-3}$\tabularnewline
\hline 
\end{tabular}

\caption{Prognostic variables of the conditionally averaged dry Euler equations.\label{tab:progs}}
\end{table}

\begin{table}[H]
\begin{tabular}{|>{\raggedright}p{0.59\textwidth}|c|c|c|}
\hline 
 & Symbol & Value & Units\tabularnewline
\hline 
\hline 
Rotation of geometry (or planet) & $\bm{\Omega}$ &  & $\text{s}^{-1}$\tabularnewline
\hline 
Reference pressure & $p_{0}$ & $10^{5}$ & Pa\tabularnewline
\hline 
Dry air $\kappa$ & $\kappa$ & $R_{a}/c_{pa}$ & -\tabularnewline
\hline 
Acceleration due to gravity & $g$ & 9.81 & $\text{m s}^{-2}$\tabularnewline
\hline 
Reference temperature & $T_{0}$ & 273.16 & K\tabularnewline
\hline 
Gas constant of dry air & $R$ & 287 & $\text{J kg}^{-1}\text{ K }^{-1}$\tabularnewline
\hline 
Heat capacity at constant pressure of dry air & $c_{p}$ & 1004 & $\text{J kg}^{-1}\text{ K }^{-1}$\tabularnewline
\hline 
Face area vector (normal to each face) & $\mathbf{S}$ &  & $\text{m}^{2}$\tabularnewline
\hline 
\end{tabular}

\caption{Constants of the conditionally averaged dry Euler equations.\label{tab:constants}}
\end{table}

\begin{table}[H]
\begin{tabular}{|>{\raggedright}p{0.39\textwidth}|c|c|c|}
\hline 
Diagnostic variable & Symbol & Calculated as & Units\tabularnewline
\hline 
\hline 
Velocity of partition $i$ & $\mathbf{u}_{i}$ & Equation (\ref{eq:reconU}) & $\text{m s}^{-1}$\tabularnewline
\hline 
Density of partition $i$ & $\rho_{i}$ & $\frac{p_{0}}{R\theta_{i}}\pi^{\frac{1-\kappa}{\kappa}}$ & $\text{kg m}^{-3}$\tabularnewline
\hline 
Volume fraction of partition $i$ & $\sigma_{i}$ & $\frac{(\sigma_{i}\rho_{i})}{\rho_{i}}$ & \tabularnewline
\hline 
Transfer rate from $i$ to $j$ & $S_{ij}$ & See section \ref{sec:transfers} & $\text{s}^{-1}$\tabularnewline
\hline 
Drag on $i$ from $j$ & $\mathbf{d}_{ij}$ & See section \ref{sec:transfers} & $\text{kg m}^{-2}\text{s}^{-1}$\tabularnewline
\hline 
Heat transfer from $i$ to $j$ & $H_{ij}$ & See section \ref{sec:transfers} & $\text{K s}^{-1}$\tabularnewline
\hline 
Compressibility & $\Psi$ & $\rho/\pi$ & \tabularnewline
\hline 
Temperature in $i$ & $T_{i}$ & $\theta_{i}\pi$ & K\tabularnewline
\hline 
Pressure & p & $p_{0}\pi^{1/\kappa}$ & Pa\tabularnewline
\hline 
Total density & $\rho$ & $\sum_{i}\sigma_{i}\rho_{i}$ & $\text{kg m}^{-3}$\tabularnewline
\hline 
Total potential temperature & $\theta$ & $\frac{\sum_{i}\sigma_{i}\rho_{i}\theta_{i}}{\rho}$ & K\tabularnewline
\hline 
Total mass fluxes & $F$ & $\sum_{i}F_{i}$ & $\text{kg s}^{-1}$\tabularnewline
\hline 
\end{tabular}

\caption{Diagnostic variables of the conditionally averaged dry Euler equations.\label{tab:diags}}
\end{table}

\section{Transfers between Partitions\label{sec:transfers}}

The transfers between partitions are the terms $S_{ij}$, the mass
transfer rate from partition $i$ to $j$, $\mathbf{d}_{ij}$, the
drag exerted on partition $i$ from partition $j$ and $H_{ij}$,
the heat transfer from partition $j$ to partition $i$.

\subsection{Drag Between Partitions}

Drag between partitions is based on a model for drag on rising bubbles
described by \citet{RLD+11}, assuming exactly two partitions. Assuming
that we use the total fluid density, the drag between partitions in
eqn (\ref{eq:condMom}) could be: 
\begin{equation}
\mathbf{d}_{ij}=\frac{\rho_{e}C_{D}}{r_{c}}|\mathbf{u}_{i}-\mathbf{u}_{j}|\left(\mathbf{u}_{i}-\mathbf{u}_{j}\right)
\end{equation}
where $C_{D}$ is a drag coefficient and $r_{c}$ is the radius or
length scale of a region of fluid. So that $\mathbf{d}_{ij}=-\mathbf{d}_{ji}$
we would have to use $r_{c}$ the same for each partition. We will
examine sensitivity to $C_{D}$ and $r_{c}$. $\rho_{e}$ is the density
of the fluid surrounding the bubble. We can therefore use the density
of the environmental air.

\subsection{Heat Transfer between Partitions}

The heat transfer coefficient is based on correlations reported in
Wikipedia for convection above and below hot plates in turbulent flow.
The amount of heat transferred is related to a heat transfer coefficient,
$h$. For external, turbulent flow over a horizontal plate, the heat
transfer coefficient can be modelled as
\begin{equation}
h=\frac{0.14\ k\ Ra_{L}^{\frac{1}{3}}}{L}
\end{equation}
where $L$ is the plate surface area divided by the plate perimeter,
$k$ is the thermal conductivity (we use $k=0.024\ \text{W\ m}^{-1}\text{K}^{-1}=0.024\ \text{kg}\ \text{m}\ \text{s}^{-3}\text{K}^{-1}$
for air) and $Ra_{L}$ is the Rayleigh number given by
\begin{equation}
Ra_{L}=\frac{g\beta}{\nu\alpha}\left(T_{s}-T_{\infty}\right)x^{3}
\end{equation}
where $\beta=1/T$ is the thermal expansion coefficient, $\nu=1.48\times10^{-1}\text{m}^{2}\text{s}^{-1}$
is the kinematic viscosity, $\alpha=k/(\rho c_{p})\ \text{m}^{2}\text{s}^{-1}$
is the thermal diffusivity, $T_{s}$ is the temperature of the plate,
$T_{\infty}$ is the air temperature far from the plate and $x$ is
a characteristic length scale. From these we derive:
\begin{equation}
H_{ij}=\frac{0.14\ k\ \left(T_{j}-T_{i}\right)}{\sigma_{i}\rho_{i}\ c_{p}\ L}\left(\frac{g\rho c_{p}}{\nu Tk}|T_{j}-T_{i}|\right)^{\frac{1}{3}}\ \text{kg}\ \text{s}^{-1}\ \text{K}
\end{equation}
for some length scale, $L$. In order to ensure that the same heat
is transferred in both directions, $\rho$ and $T$ are respectively
the volumetric and mass means of both partitions.

The heat transfer between partitions should be limited by the amount
of heat needed to equalised the temperatures in one time-step:
\begin{equation}
H_{ij\ \max}=\frac{\theta_{j}-\theta_{i}}{\Delta t}\frac{\sigma_{j}\rho_{j}\sigma_{i}\rho_{i}}{\sigma_{j}\rho_{j}+\sigma_{i}\rho_{i}}.
\end{equation}
However even using the maximum heat transfer between partitions, the
model is still unstable.

It is essential to ensure that:
\begin{equation}
\sigma_{i}\rho_{i}H_{ij}=-\sigma_{j}\rho_{j}H_{ji}
\end{equation}
for energy conservation.

\subsection{Diffusion of partition volume fraction, $\sigma$\label{subsec:diffusionTransfer}}

Changes in gradient of $\sigma$ (and in particular oscillations)
can be reduced by diffusion between partitions through the mass transfer
terms. We introduce diffusion of $\sigma$ in such a way that total
mass is not diffused and so that the transfer from $i$ to $j$ is
always positive:
\begin{equation}
\sigma_{i}\rho_{i}S_{ij}=\frac{K_{\sigma}}{2}\max\left(\nabla^{2}\left(\sigma_{j}\rho_{j}-\sigma_{i}\rho_{i}\right),\ 0\right)\label{eq:diffusionTransfer}
\end{equation}
 The equation for $\sigma_{i}\rho_{i}$ becomes:
\begin{eqnarray}
\frac{\partial\sigma_{i}\rho_{i}}{\partial t}+\nabla\cdot(\sigma_{i}\rho_{i}\mathbf{u}_{i}) & = & K_{\sigma}\sum_{j\ne i}\left(\nabla^{2}\left(\sigma_{i}\rho_{i}-\sigma_{j}\rho_{j}\right)\right).\label{eq:sigmaRhoWithTransfer}
\end{eqnarray}
If eqn (\ref{eq:sigmaRhoWithTransfer}) is solved with operator splitting
and the right hand side treated explicitly, positivity of $\sigma_{i}\rho_{i}$
is not violated by the $K_{\sigma}\nabla^{2}\sigma_{i}\rho_{i}$ term
as long as the ratio $K_{\sigma}\Delta t/\Delta x^{2}$ is small enough.
In order to prevent the $K_{\sigma}\nabla^{2}\sigma_{j}\rho_{j}$
term from removing too much $\sigma_{i}\rho_{i}$ in one time-step
it can be limited, depending on the time-stepping scheme. For example:
\begin{equation}
\sigma_{i}\rho_{i}S_{ij}=\min\left(\sigma_{i}\rho_{i}S_{ij},\ \sigma_{i}\rho_{i}/\Delta t\right)\label{eq:limitDiffusionTransfer}
\end{equation}

\section{Transport Equations in Terms of Fluxes}

For conservation and consistency, the same fluxes, $F_{i}=\sigma_{i}\rho_{i}\mathbf{u}_{i}\cdot\mathbf{S}$,
are used to update the density, potential temperature and velocity
in each partition. For brevity, we write the divergence operator in
semi-discretised form or in discretised form using Gauss's divergence
theorem:
\begin{equation}
\nabla\cdot(\sigma_{i}\rho_{i}\mathbf{u}_{i}\Upsilon)=\nabla\cdot(F_{i}\Upsilon)=\frac{1}{V}\sum_{f}F_{i}\Upsilon_{f}
\end{equation}
where $V$ is the volume of a cell, $f$ is over each face of a cell
and $\Upsilon$ is a variable with $\Upsilon_{f}$ being the value
of the variable at the face $f$. Thus we can re-write the transport
equations in terms of $F_{i}$:
\begin{eqnarray}
\frac{\partial F_{i}}{\partial t}+\nabla\cdot\left(F_{i}\mathbf{u}_{i}\right) & = & -2\sigma_{i}\rho_{i}(\boldsymbol{\Omega}\times\mathbf{u}_{i})\cdot\mathbf{S}-\sigma_{i}\rho_{i}c_{p}\theta_{i}\nabla_{S}\pi+\sigma_{i}\rho_{i}\mathbf{g}\cdot\mathbf{S}\label{eq:condMomFlux}\\
 & + & \sum_{j\ne i}\left(F_{j}S_{ji}-F_{i}S_{ij}-\sigma_{i}\sigma_{j}\mathbf{d}_{ij}\cdot\mathbf{S}\right)\nonumber \\
\frac{\partial\sigma_{i}\rho_{i}}{\partial t}+\nabla\cdot F_{i} & = & \sum_{j\ne i}\left(\sigma_{j}\rho_{j}S_{ji}-\sigma_{i}\rho_{i}S_{ij}\right)\label{eq:condContFlux}\\
\frac{\partial\sigma_{i}\rho_{i}\theta_{i}}{\partial t}+\nabla\cdot\left(F_{i}\theta_{i}\right) & = & \sum_{j\ne i}\left(\sigma_{j}\rho_{j}\theta_{j}S_{ji}-\sigma_{i}\rho_{i}\theta_{i}S_{ij}-\sigma_{i}\rho_{i}H_{ij}\right)\label{eq:condThetaFlux}
\end{eqnarray}
Before the construction and numerical solution of the Helmholtz equation
to update $\pi$ (section \ref{sec:helm}), new values of $\sigma_{i}\rho_{i}$
are predicted using eqn (\ref{eq:condContFlux}) and values of $\theta_{i}$
are predicted using eqn (\ref{eq:condThetaFlux}) and using the new
values of $\sigma_{i}\rho_{i}$. These initial predictions are corrected
in an iterative outer predictor-corrector loop.

The fluxes, $F_{i}$, in equation (\ref{eq:condContFlux}) can lead
to unbounded solutions for $\sigma_{i}\rho_{i}$. In order to avoid
this, $F_{i}$ is rewritten as:
\begin{eqnarray*}
F_{i} & = & \sigma_{i}F-\sigma_{i}\sum_{j}F_{j}+\sigma_{i}\frac{F_{i}}{\sigma_{i}}\\
 & = & \sigma_{i}F+\sigma_{i}\frac{F_{i}}{\sigma_{i}}-\sigma_{i}(1-\sigma_{i})\sum_{j}\frac{F_{j}}{(1-\sigma_{i})}\\
 & = & \sigma_{i}F+\sigma_{i}(1-\sigma_{i})\left(\frac{F_{i}}{\sigma_{i}}-\frac{\sum_{j\ne i}F_{j}}{\sum_{j\ne i}\sigma_{j}}\right)
\end{eqnarray*}
This doesn't work. It only makes things worse.

\section{Constructing the Helmholtz Equation\label{sec:helm}}

The Helmholtz equation for $\pi$ is constructed from the continuity
equation by substituting
\begin{equation}
\rho=\Psi\pi\label{eq:rhoComp}
\end{equation}
into the rate of change of density in eqn (\ref{eq:condContFlux})
\begin{equation}
\frac{\partial\Psi\pi}{\partial t}+\nabla\cdot F=0.\label{eq:contForPi}
\end{equation}
$\Psi$ is the compressibility which is calculated as $\Psi=\rho/\pi$.
$\rho$ is calculated from the sum of the partitions, $\rho=\sum_{i}\sigma_{i}\rho_{i}$
where each $\sigma_{i}\rho_{i}$ is updated from eqn (\ref{eq:condContFlux})
and $\pi$ is calculated from the equation of state (\ref{eq:condState})
using updated values of $\theta_{i}$. The fluxes, $F_{i}$, are updated
using the momentum equation (\ref{eq:condMomFlux}):
\begin{equation}
F_{i}^{n+1}=F_{i}^{\prime}-\alpha\Delta tc_{p}\sigma_{i}\rho_{i}\theta_{i}\nabla_{S}\pi^{n+1}\label{eq:newFluxi}
\end{equation}
where
\begin{equation}
\nabla_{S}\pi=\mathbf{S}\cdot\nabla\pi,
\end{equation}
$\alpha$ is the off-centering parameter for the time-stepping and
the explicit component of $F_{i}$ is $F_{i}^{\prime}$ calculated
from the momentum equation (\ref{eq:condMomFlux}) without the new
value of the pressure gradient term
\begin{eqnarray*}
F_{i}^{\prime}=F_{i}^{n}- & (1-\alpha)\Delta t & \left\{ \nabla\cdot\left(F_{i}\mathbf{u}_{i}\right)+2\sigma_{i}\rho_{i}\boldsymbol{\Omega}\times\mathbf{u}_{i}+c_{p}\sigma_{i}\rho_{i}\theta_{i}\nabla\pi-\sigma_{i}\rho_{i}\mathbf{g}\right\} ^{n}\cdot\mathbf{S}\\
- & \alpha\Delta t & \left\{ \nabla\cdot\left(F_{i}\mathbf{u}_{i}\right)+2\sigma_{i}\rho_{i}\boldsymbol{\Omega}\times\mathbf{u}_{i}-\sigma_{i}\rho_{i}\mathbf{g}\right\} ^{\ell}\cdot\mathbf{S}\\
 & + & \Delta t\sum_{j\ne i}\left\{ \sigma_{j}\rho_{j}\mathbf{u}_{j}S_{ji}-\sigma_{i}\rho_{i}\mathbf{u}_{i}S_{ij}-\sigma_{i}\sigma_{j}\mathbf{d}_{ij}\right\} ^{n+1/2}
\end{eqnarray*}
where superscripts $n$ and $n+1$ specify the time-level and superscript
$\ell$ means the most up to date value of time level $n+1$ calculated
explicitly. This assumes predictor-corrector time-stepping such as
multi-stage RK2. Equation (\ref{eq:newFluxi}) is then substituted
into equation (\ref{eq:contForPi}) to create the Helmholtz equation
for $\pi$:
\begin{equation}
\frac{\partial\Psi\pi}{\partial t}+\nabla\cdot\sum_{i}F_{i}^{\prime}-\alpha\Delta t\sum_{i}\sigma_{i}c_{p}\rho\theta\nabla_{S}\pi^{n+1}=0.\label{eq:Helmholtz}
\end{equation}
The solution for $\pi^{n+1}$ is then substituted back into eqn (\ref{eq:newFluxi})
to calculate each $F_{i}^{n+1}$. The coefficient $\sum_{i}\sigma_{i}$
is retained in the Helmholtz equation because it may not exactly sum
to one. 

\section{Upwinding of $\sigma_{i}$}

Equation (\ref{eq:newFluxi}) gives the flux of $\sigma_{i}\rho_{i}$
and so the continuity equation for partition $i$ (eqn \ref{eq:condContFlux})
can update $\sigma_{i}\rho_{i}$ without further numerical approximation.
However this leads to unbounded solutions. Instead we need to advect
$\sigma_{i}$ using the flux, $F_{i}/\sigma_{i}$ so that upwinding
can be used in the advection to avoid spurious undershoots and overshoots.
This will be attempted by calculating $F_{i}/\sigma_{i}$ using linear
interpolation to interpolate $\sigma_{i}$ from cell centres onto
faces (with $F_{i}$). Then upwinding of $\sigma_{i}$ can be used
in the continuity equation:
\begin{equation}
\frac{\partial\sigma_{i}\rho_{i}}{\partial t}+\nabla\cdot\left(\frac{F_{i}}{\sigma_{i\text{lin}}}\sigma_{i\text{up}}\right)=\sum_{j\ne i}\left(\sigma_{j}\rho_{j}S_{ji}-\sigma_{i}\rho_{i}S_{ij}\right)
\end{equation}

\section{Updating the partition fraction, $\sigma_{i}$}

The above solution of the Helmholtz equation updates $\pi$ and the
back substitution gives fluxes $F$ which satisfy eqn (\ref{eq:contForPi})
and partition fluxes $F_{i}$ which satisfy $\sum_{i}F_{i}=F$ given
$\rho\theta=\sum_{i}\sigma_{i}\rho_{i}\theta_{i}$. We now need to
update $\sigma_{i}$ so that eqn (\ref{eq:condContFlux}) is satisfied
in each partition, the equation of state (\ref{eq:condStatei}) is
satisfied in each partition using the same pressure and satisfying
$\sum_{i}\sigma_{i}=1$. In order to achieve this, the joint variables
$(\sigma_{i}\rho_{i})$ are first updated using eqn (\ref{eq:condContFlux}).
The density in each partition is calculated so that the equation of
state holds in each partition:
\begin{equation}
\rho_{i}=\frac{p_{0}\pi^{\frac{1-\kappa}{\kappa}}}{R\theta_{i}}=\frac{\rho\theta}{\theta_{i}}\label{eq:stateForRhoi}
\end{equation}
and finally $\sigma_{i}=\frac{(\sigma_{i}\rho_{i})}{\rho_{i}}$. This
will automatically guarantee $\sum_{i}\sigma_{i}=1$ since if we substitute
eqn (\ref{eq:stateForRhoi}) into the definition of $\sigma_{i}$
we get:
\begin{equation}
\sum_{i}\sigma_{i}=\sum_{i}\frac{(\sigma_{i}\rho_{i})}{\rho_{i}}=\sum_{i}\frac{(\sigma_{i}\rho_{i})}{\rho\theta/\theta_{i}}=\frac{\rho\theta}{\rho\theta}=1.
\end{equation}

\section{Spatial Discretisation}

A staggered finite volume discretisation is used. The velocity at
cell faces for each partition is reconstructed from the face mass
flux and the partition density:
\begin{equation}
\mathbf{u}_{i}=\left(\left(\sum_{\text{faces}}\mathbf{S}\mathbf{S}^{T}\right)^{-1}\sum_{\text{faces}}F_{i}\mathbf{S}\right)_{f}\bigg/\left(\sigma_{i}\rho_{i}\right)_{f}\label{eq:reconU}
\end{equation}
where operator $\left(\right)_{f}$ means interpolation from cell
centre values onto face values and $\mathbf{S}$ is the face area
vector (outward pointing normal to each face with magnitude of the
face area. For a uniform velocity field, this reconstruction formula
reproduces exactly the definition of the face flux, $F_{i}=\sigma_{i}\rho_{i}\mathbf{u}_{i}\cdot\mathbf{S}$
given in table \ref{tab:progs}. The summations are over all of the
faces of a cell.

The velocity field in each partition is updated after updating $\sigma_{i}\rho_{i}$,
$\rho_{i}$ and $\sigma_{i}$.

A multi-dimensional, method of lines, linear upwind advection scheme
is used for advecting species concentrations, velocity and potential
temperature.

\section{Defining Boundary Conditions}

Boundary conditions must be set for the variables $\pi$ and $\partial F_{i}/\partial t$.
Fixed flux boundary conditions are applied by ensuring that $\partial F_{i}/\partial t$
is zero at these boundaries. $\pi$ is set at fixed flux boundaries
by assuming hydrostatic pressure in the direction normal to each boundary:
\begin{equation}
c_{p}\theta_{\rho}\nabla_{S}\pi=\mathbf{g}\cdot\mathbf{S}.
\end{equation}

\section{Advective Form of the equations}

The conditionally averaged equations in advective form are:
\begin{eqnarray}
\frac{\partial\mathbf{u}_{i}}{\partial t}+\mathbf{u}_{i}\cdot\nabla\mathbf{u}_{i} & = & -2\boldsymbol{\Omega}\times\mathbf{u}_{i}-c_{p}\theta_{i}\nabla\pi+\mathbf{g}\label{eq:condMomAdv}\\
 & + & \sum_{j\ne i}\left(\frac{\sigma_{j}\rho_{j}}{\sigma_{i}\rho_{i}}S_{ji}(\mathbf{u}_{j}-\mathbf{u}_{i})-\mathbf{D}_{ij}\right)\\
\frac{\partial\sigma_{i}\rho_{i}}{\partial t}+\nabla\cdot(\sigma_{i}\rho_{i}\mathbf{u}_{i}) & = & \sum_{j\ne i}\left(\sigma_{j}\rho_{j}S_{ji}-\sigma_{i}\rho_{i}S_{ij}\right)\label{eq:condCont-1}\\
\frac{\partial\theta_{i}}{\partial t}+\mathbf{u}_{i}\cdot\nabla\theta_{i} & = & \sum_{j\ne i}\left(\frac{\sigma_{j}\rho_{j}}{\sigma_{i}\rho_{i}}S_{ji}(\theta_{j}-\theta_{i})-H_{ij}\right)\label{eq:condThetaAdv}
\end{eqnarray}
where $\mathbf{D}_{ij}$ is now the drag per unit mass. The Helmholtz
equation for $\pi$ is constructed in a similar way, substituting
\begin{equation}
\rho=\Psi\pi\label{eq:rhoComp-1}
\end{equation}
into the rate of change of density in eqn (\ref{eq:condContFlux})
\begin{equation}
\frac{\partial\Psi\pi}{\partial t}+\nabla\cdot F=0.\label{eq:contForPi-1}
\end{equation}
$\Psi$ is the compressibility which is calculated as $\Psi=\rho/\pi$.
$\rho$ is calculated from the sum of the partitions, $\rho=\sum_{i}\sigma_{i}\rho_{i}$
where each $\sigma_{i}\rho_{i}$ is updated from eqn (\ref{eq:condContFlux})
and $\pi$ is calculated from the equation of state (\ref{eq:condState})
using updated values of $\theta_{i}$. The fluxes, $F_{i}$, are updated
using the momentum equation (\ref{eq:condMomAdv}):
\begin{equation}
F_{i}^{n+1}=F_{i}^{\prime}-\alpha\Delta tc_{p}\left(\sigma_{i}\rho_{i}\theta_{i}\right)^{\ell}\nabla_{S}\pi^{n+1}\label{eq:newFluxi-1}
\end{equation}
where
\begin{equation}
\nabla_{S}\pi=\mathbf{S}\cdot\nabla\pi,
\end{equation}
$\alpha$ is the off-centering parameter for the time-stepping and
the explicit component of $F_{i}$ is $F_{i}^{\prime}$ calculated
from the momentum equation (\ref{eq:condThetaAdv}) without the new
value of the pressure gradient term
\begin{eqnarray*}
F_{i}^{\prime}= & \left(\sigma_{i}\rho_{i}\right)^{n+1}S\cdot\biggl\{\mathbf{u}_{i}^{n}+\Delta t\mathbf{g} & -(1-\alpha)\Delta t\left(\mathbf{u}_{i}\cdot\nabla\mathbf{u}_{i}+2\boldsymbol{\Omega}\times\mathbf{u}_{i}+c_{p}\theta_{i}\nabla\pi\right)^{n}\\
 &  & -\alpha\Delta t\left(\mathbf{u}_{i}\cdot\nabla\mathbf{u}_{i}+2\boldsymbol{\Omega}\times\mathbf{u}_{i}\right)^{\ell}\biggr\}
\end{eqnarray*}
where superscripts $n$ and $n+1$ specify the time-level and superscript
$\ell$ means the most up to date value of time level $n+1$ calculated
explicitly. This assumes predictor-corrector time-stepping such as
multi-stage RK2. Equation (\ref{eq:newFluxi-1}) is then substituted
into equation (\ref{eq:contForPi-1}) to create the Helmholtz equation
for $\pi$:
\begin{equation}
\frac{\partial\Psi\pi}{\partial t}+\nabla\cdot\sum_{i}F_{i}^{\prime}-\alpha\Delta t\sum_{i}\sigma_{i}c_{p}\rho\theta\nabla_{S}\pi^{n+1}=0.\label{eq:Helmholtz-1}
\end{equation}
The solution for $\pi^{n+1}$ is then substituted back into eqn (\ref{eq:newFluxi-1})
to calculate each $F_{i}^{n+1}$. The coefficient $\sum_{i}\sigma_{i}$
is retained in the Helmholtz equation because it may not exactly sum
to one. 

For the drag, we will try the form:
\begin{equation}
\mathbf{D}_{ij}=\sigma_{j}\frac{C_{D}}{r_{c}}\frac{\bar{\rho}}{\rho_{i}}|\mathbf{u}_{i}-\mathbf{u}_{j}|\left(\mathbf{u}_{i}-\mathbf{u}_{j}\right).
\end{equation}
As $\sigma_{i}$ becomes small in any partition, we need $r_{c}$
to become small which increases the drag between partitions. If we
assume a maximum and minimum could radius, $r_{\max}$ and $r_{\min}$,
then the cloud radius can take the form
\begin{equation}
r_{c}=\max\left(r_{\min},\ r_{\max}\prod_{i}\sigma_{i}\right).
\end{equation}
The term $\frac{\sigma_{j}\rho_{j}}{\sigma_{i}\rho_{i}}S_{ji}$ appears
on the right hand side of equations (\ref{eq:condMomAdv}) and (\ref{eq:condThetaAdv})
and so these source terms will diverge when $\sigma_{i}\rho_{i}\rightarrow0$.
Therefore this term needs to be limited:
\begin{equation}
T_{ji}=\frac{\sigma_{j}\rho_{j}}{\rho_{i}\max(\sigma_{i},\ \varepsilon)}S_{ji}
\end{equation}
for some small $\varepsilon$. However these source terms can still
become very large for small $\sigma_{i}$ and so should be treated
implicitly with operator splitting so that equations (\ref{eq:condMomAdv})
and (\ref{eq:condThetaAdv}) are solved with a zero right hand side
and the solutions are $\mathbf{u}_{i}^{\prime}$ and $\theta_{i}^{\prime}$.
Then equations (\ref{eq:condMomAdv}) and (\ref{eq:condThetaAdv})
can be re-written (without drag or heat transfer) as:
\begin{eqnarray}
\mathbf{u}_{i}^{n+1} & = & \mathbf{u}_{i}^{\prime}+\Delta t\sum_{j\ne i}\left(T_{ji}(\mathbf{u}_{j}^{n+1}-\mathbf{u}_{i}^{n+1})\right)\label{eq:condMomAdv-1}\\
\theta_{i}^{n+1} & = & \theta_{i}^{\prime}+\Delta t\sum_{j\ne i}\left(T_{ji}(\theta_{j}^{n+1}-\theta_{i}^{n+1})\right).\label{eq:condThetaAdv-1}
\end{eqnarray}
If $i,j=1,2$ then these can be re-arranged as:
\begin{eqnarray}
\left(\begin{array}{cc}
1+\Delta t\ T_{21} & -\Delta t\ T_{21}\\
-\Delta t\ T_{12} & 1+\Delta t\ T_{12}
\end{array}\right)\left(\begin{array}{c}
\mathbf{u}_{1}^{n+1}\\
\mathbf{u}_{2}^{n+1}
\end{array}\right) & = & \left(\begin{array}{c}
\mathbf{u}_{1}^{\prime}\\
\mathbf{u}_{2}^{\prime}
\end{array}\right)\\
\left(\begin{array}{cc}
1+\Delta t\ T_{21} & -\Delta t\ T_{21}\\
-\Delta t\ T_{12} & 1+\Delta t\ T_{12}
\end{array}\right)\left(\begin{array}{c}
\theta_{1}^{n+1}\\
\theta_{2}^{n+1}
\end{array}\right) & = & \left(\begin{array}{c}
\theta_{1}^{\prime}\\
\theta_{2}^{\prime}
\end{array}\right)
\end{eqnarray}
which can be solved implicitly for $\mathbf{u}_{i}$ and $\theta_{i}$
with construction of only local two by two matrices, for example:
\begin{equation}
\left(\begin{array}{c}
\theta_{1}^{n+1}\\
\theta_{2}^{n+1}
\end{array}\right)=\frac{1}{1+\Delta t\ T_{21}+\Delta t\ T_{12}}\left(\begin{array}{cc}
1+\Delta t\ T_{12} & \Delta t\ T_{21}\\
\Delta t\ T_{12} & 1+\Delta t\ T_{21}
\end{array}\right)\left(\begin{array}{c}
\theta_{1}^{\prime}\\
\theta_{2}^{\prime}
\end{array}\right).
\end{equation}
Alternatively, to limit storage, $\theta_{1}^{n+1}$ and $\theta_{2}^{n+1}$
can be calculated as:
\begin{eqnarray*}
\theta_{1}^{n+1} & = & \frac{1}{1+\Delta t\ T_{21}+\Delta t\ T_{12}}\left(\left(1+\Delta t\ T_{12}\right)\theta_{1}^{\prime}+\Delta t\ T_{21}\theta_{2}^{\prime}\right)\\
\theta_{2}^{n+1} & = & \frac{\theta_{2}^{\prime}+\Delta t\ T_{12}\theta_{1}^{n+1}}{1+\Delta T_{12}}.
\end{eqnarray*}

\section{Mass Transfer Between Partitions based on vertical velocity}

In order to estimate the mass transfer between partitions we need
to know the area of the interface between the partitions, $A=A_{ij}=A_{ji}$.
The the mass transfer can be defined as
\begin{eqnarray}
M_{12} & = & \sigma_{1}\rho_{1}S_{12}=\begin{cases}
\rho_{1}w_{1}A & \ \text{when}\ w_{1}>0\\
0 & \ \text{otherwise}
\end{cases}\\
M_{21} & = & \sigma_{2}\rho_{2}S_{21}=\begin{cases}
-\rho_{2}w_{2}A & \ \text{when}\ w_{2}<0\\
0 & \ \text{otherwise}.
\end{cases}
\end{eqnarray}
For an arbitrary shaped cell, the area $A$ can be estimated as
\begin{equation}
A=\frac{1}{2}\sum_{f\in\text{faces}}\biggl|\mathbf{S}_{f}\cdot\widehat{\nabla\sigma}\biggr|\approx\frac{1}{2}\frac{\sum_{f\in\text{faces}}\biggl|\nabla_{S}\sigma\biggr|}{|\nabla\sigma|}
\end{equation}
for either $\sigma$ where $\widehat{\nabla\sigma}=\frac{\nabla\sigma}{|\nabla\sigma|}$
and $\nabla_{S}\sigma=\mathbf{S}_{f}\cdot\nabla\sigma$. But if $\nabla\sigma=0$
then there can be no transfer. 

\section{Mass Transfers Between Partitions based on horizontal divergence}

We can calculate the horizontal divergence of a vector $\mathbf{v}$
as
\begin{equation}
\nabla_{h}\cdot\mathbf{v}=\nabla\cdot\left(\mathbf{v}-(\mathbf{v}\cdot\hat{\mathbf{g}})\hat{\mathbf{g}}\right).
\end{equation}
The mass transfer can be calculated from the horizontal mass divergence
as:
\begin{eqnarray}
\sigma_{1}\rho_{1}S_{12} & = & \begin{cases}
-\nabla_{h}\cdot(\sigma_{1}\rho_{1}\mathbf{u}_{1}) & \ \text{if}\ \nabla_{h}\cdot(\sigma_{1}\rho_{1}\mathbf{u}_{1})<0\ \text{and}\ \mathbf{u}_{1}\cdot\mathbf{g}<0\\
0 & \ \text{otherwise}
\end{cases}\\
\sigma_{2}\rho_{2}S_{21} & = & \begin{cases}
\nabla_{h}\cdot(\sigma_{2}\rho_{2}\mathbf{u}_{2}) & \ \text{if}\ \nabla_{h}\cdot(\sigma_{2}\rho_{2}\mathbf{u}_{2})>0\ \text{and}\ \mathbf{u}_{2}\cdot\mathbf{g}>0\\
0 & \ \text{otherwise}
\end{cases}.
\end{eqnarray}
A problem is that mass is transferred to the buoyant partition rather
slowly.

\section{Mass Transfers Between Partitions based on $\nabla^{2}\theta$}

Section \ref{subsec:diffusionTransfer} defined mass transfer terms
between partitions based on diffusion which will smooth out the $\sigma$
fields. However mass transfers are also needed in order to direct
buoyant or convecting air into the unstable partition. In the absence
of information on sub-grid scale variability, we assume that there
is no sub-grid scale variability and so we want rising motion or buoyant
air to be in the buoyant partition. Assuming that the stable partition
is number 1 and the buoyant partition is number 2, to send buoyant
air into the buoyant partition if it is warmer than it's surrounding,
ie if $\nabla^{2}\theta<0$. The rate of sending air into the buoyant
partition should be dependent on the size of $\nabla^{2}\theta<0$.
We could choose a frequency $K_{\theta}\frac{\nabla^{2}\theta}{\theta}$
where $K_{\theta}$ is a diffusivity in $\text{m}^{2}\text{s}^{-1}$.
\begin{eqnarray}
S_{12} & = & \begin{cases}
\max\left(-K_{\theta}\frac{\nabla^{2}\theta_{1}}{\theta_{1}},\ 1/\Delta t\right) & \ \text{when}\ \nabla^{2}\theta_{1}<0\\
0 & \ \text{otherwise}
\end{cases}\\
S_{21} & = & \begin{cases}
\max\left(K_{\theta}\frac{\nabla^{2}\theta_{2}}{\theta_{2}},\ 1/\Delta t\right) & \ \text{when}\ \nabla^{2}\theta_{2}>0\\
0 & \ \text{otherwise}
\end{cases}.
\end{eqnarray}
This frequency seems to work ok but however you start it, both partitions
end up with the same properties. We need somehow to transfer the excess
$\theta$ into the other partition so as to remove variations in $\theta$.
Ie the mass transferred sucks heat out of where it came from.

These need to be combined with equations (\ref{eq:diffusionTransfer})
and (\ref{eq:limitDiffusionTransfer}) so that diffusion can be used
in combination:
\begin{eqnarray}
\sigma_{1}\rho_{1}S_{12} & = & \frac{K_{\sigma}}{2}\max\left(\nabla^{2}\left(\sigma_{2}\rho_{2}-\sigma_{1}\rho_{1}\right),\ 0\right)+\begin{cases}
-K_{\theta}\frac{\nabla^{2}\theta_{1}}{\theta_{1}} & \ \text{when}\ \nabla^{2}\theta_{1}<0\\
0 & \ \text{otherwise}
\end{cases}\\
\sigma_{2}\rho_{2}S_{21} & = & \frac{K_{\sigma}}{2}\max\left(\nabla^{2}\left(\sigma_{1}\rho_{1}-\sigma_{2}\rho_{2}\right),\ 0\right)+\begin{cases}
K_{\theta}\frac{\nabla^{2}\theta_{2}}{\theta_{2}} & \ \text{when}\ \nabla^{2}\theta_{2}>0\\
0 & \ \text{otherwise}
\end{cases}.
\end{eqnarray}
and then limited as before:
\begin{equation}
\sigma_{i}\rho_{i}S_{ij}=\min\left(\sigma_{i}\rho_{i}S_{ij},\ \sigma_{i}\rho_{i}/\Delta t\right)\label{eq:limitDiffusionTransfer-1}
\end{equation}

\section{Mass Transfer based on Partition Divergence}

There is nothing to control the partition divergence, $\nabla\cdot\mathbf{u}_{i}$
so it can become large which leads to large changes in $\sigma_{i}\rho_{i}$.
We want $\mathbf{u}_{i}$ to move $\sigma_{i}\rho_{i}$ around but
we the partition divergence is a problem. How about getting partition
divergence to move fluid into the other partition. Considering just
two partitions, this could be achieved by:
\begin{equation}
\sigma_{j}\rho_{j}S_{ji}-\sigma_{i}\rho_{i}S_{ij}=\frac{1}{2}\left(\sigma_{i}\rho_{i}\nabla\cdot\mathbf{u}_{i}-\sigma_{j}\rho_{j}\nabla\cdot\mathbf{u}_{j}.\right)
\end{equation}
In order to separate this into strictly positive source terms we can
use:
\begin{eqnarray*}
\sigma_{j}\rho_{j}S_{ji} & = & \frac{1}{2}\max\left(\sigma_{i}\rho_{i}\nabla\cdot\mathbf{u}_{i}-\sigma_{j}\rho_{j}\nabla\cdot\mathbf{u}_{j},\ 0\right).
\end{eqnarray*}
This is equivalent to the transport equation for $\sigma_{i}\rho_{i}$
being: 
\begin{equation}
\frac{\partial\sigma_{i}\rho_{i}}{\partial t}+\mathbf{u}_{i}\cdot\nabla(\sigma_{i}\rho_{i})=-\frac{1}{2}\overline{\rho}\nabla\cdot\overline{\mathbf{u}}
\end{equation}
where $\overline{\rho}\nabla\cdot\overline{\mathbf{u}}=\sum_{i}\sigma_{i}\rho_{i}\nabla\cdot\mathbf{u}_{i}$,
satisfying the criteria that, in the absence of global divergence,
advection of $\sigma_{i}\rho_{i}$ is bounded. This transfer term
is formulated to stabilise the equations rather than to represent
buoyant convection.

\section{Internal Energy Conservation on Mass Transfer}

Internal energy should be conserved on mass transfer and we would
like our numerical method to have the same property. This implies
that we would like:
\begin{equation}
\sum_{i}(\sigma_{i}\rho_{i}\theta_{i})^{n+1}=\sum_{i}(\sigma_{i}\rho_{i}\theta_{i})^{n}
\end{equation}
Considering only the transfer terms we use an explicit update of $\sigma_{i}\rho_{i}$:
\begin{equation}
(\sigma_{i}\rho_{i})^{n+1}=(\sigma_{i}\rho_{i})^{n}+\Delta t\left((\sigma_{j}\rho_{j})^{n}S_{ji}^{n}-(\sigma_{i}\rho_{i})^{n}S_{ij}^{n}\right)
\end{equation}
and an implicit update of $\theta_{i}$:
\begin{eqnarray*}
\theta_{0}^{n+1} & = & \frac{\left(1+T_{01}\right)\theta_{0}^{n}+T_{10}\theta_{1}^{n}}{1+T_{10}+T_{01}}\\
\theta_{1}^{n+1} & = & \frac{\theta_{1}^{n}+T_{01}\theta_{0}^{n+1}}{1+T_{01}}
\end{eqnarray*}
where $T_{ij}=\Delta t\ \frac{\sigma_{i}\rho_{i}}{\sigma_{j}\rho_{j}}S_{ij}$.
Without loss of generality we can consider the case where $S_{ji}=0$
and $i=0$, $j=1$ and use the shorthand $h_{i}=\sigma_{i}\rho_{i}$
giving:
\begin{eqnarray*}
h_{o}^{n+1} & = & h_{0}^{n}\left(1-\Delta t\ S_{01}\right)\\
h_{1}^{n+1} & = & h_{1}^{n}+h_{o}^{n}\Delta t\ S_{01}\\
\theta_{0}^{n+1} & = & \theta_{0}^{n}\\
\theta_{1}^{n+1} & = & \frac{\theta_{1}^{n}+\Delta t\ \frac{h_{0}}{h_{1}}S_{01}\theta_{0}^{n}}{1+\Delta t\ \frac{h_{0}}{h_{1}}S_{01}}
\end{eqnarray*}
So the energy conservation error is:
\begin{eqnarray*}
\sum_{i}(\sigma_{i}\rho_{i}\theta_{i})^{n+1}-\sum_{i}(\sigma_{i}\rho_{i}\theta_{i})^{n} & = & h_{0}^{n}\left(1-\Delta t\ S_{01}\right)\theta_{0}^{n}+\left(h_{1}^{n}+h_{o}^{n}\Delta t\ S_{01}\right)\frac{\theta_{1}^{n}+\Delta t\ \frac{h_{0}}{h_{1}}S_{01}\theta_{0}^{n}}{1+\Delta t\ \frac{h_{0}}{h_{1}}S_{01}}\\
 & - & h_{0}^{n}\theta_{0}^{n}-h_{1}^{n}\theta_{1}^{n}\\
 & =- & \Delta t\ S_{01}h_{0}^{n}\theta_{0}^{n}+h_{1}^{n}\Delta t\ \frac{h_{0}}{h_{1}}S_{01}\theta_{0}^{n}\\
 & = & \Delta t\ S_{01}\theta_{0}^{n}\left(-h_{0}^{n}+h_{1}^{n}\frac{h_{0}}{h_{1}}\right)
\end{eqnarray*}
So energy conservation is achieved as long as we use old values of
$\sigma_{i}\rho_{i}$ in the calculation of $T_{ij}$.

\section{Temperature of the Mass Transferred based on swapping}

Equation (\ref{eq:condThetaAdv}) assumes that the mass transferred
has the temperature of the partition that it is leaving. Thus there
is nothing in these equations that forces the temperatures in each
partition away from each other. If instead we assume that the temperature
of the mass transferred from partition 1 to 2 has the higher of the
two temperatures and vice-verca then partition 2 will always tend
to be warmer, consistent with it rising. Then the flux form of the
$\theta$ equations become: 
\begin{eqnarray}
\frac{\partial\sigma_{1}\rho_{1}\theta_{1}}{\partial t}+\nabla\cdot\sigma_{1}\rho_{1}\theta_{1}\mathbf{u}_{1} & = & \sigma_{2}\rho_{2}S_{21}\max(\theta_{1},\theta_{2})-\sigma_{1}\rho_{1}S_{12}\min(\theta_{1},\theta_{2})\\
\frac{\partial\sigma_{2}\rho_{2}\theta_{2}}{\partial t}+\nabla\cdot\sigma_{2}\rho_{2}\theta_{2}\mathbf{u}_{2} & = & \sigma_{1}\rho_{1}S_{12}\min(\theta_{1},\theta_{2})-\sigma_{2}\rho_{2}S_{21}\max(\theta_{1},\theta_{2}).
\end{eqnarray}
With the continuity equation, this leads to the advective form:
\begin{eqnarray*}
\frac{\partial\theta_{1}}{\partial t}+\mathbf{u}_{1}\cdot\nabla\theta_{1} & = & \frac{\sigma_{2}\rho_{2}}{\sigma_{1}\rho_{1}}S_{21}\left(\max(\theta_{1},\theta_{2})-\theta_{1}\right)-S_{12}\left(\min(\theta_{1},\theta_{2})-\theta_{1}\right)\\
\frac{\partial\theta_{2}}{\partial t}+\mathbf{u}_{2}\cdot\nabla\theta_{2} & = & \frac{\sigma_{1}\rho_{1}}{\sigma_{2}\rho_{2}}S_{12}\left(\min(\theta_{1},\theta_{2})-\theta_{2}\right)-S_{21}\left(\max(\theta_{1},\theta_{2})-\theta_{2}\right)
\end{eqnarray*}
It is going to be much more difficult to make this implicit due to
the severe non-linearity of the min and max functions. 

\section{Swapping Temperature through the Heat Transfer Term}

If $\theta_{1}>\theta_{2}$ and $S_{12}>0$ then we want the temperatures
to swap (or tend towards swapping). This can be achieved by:
\begin{eqnarray*}
\sigma_{1}\rho_{1}H_{12} & = & M_{12}\max(\theta_{1}-\theta_{2},0)-M_{21}\min(\theta_{2}-\theta_{1},0)
\end{eqnarray*}
and $\sigma_{2}\rho_{2}H_{21}=-\sigma_{1}\rho_{1}H_{12}$ for energy
conservation. This will require some limiting for stability:
\begin{equation}
\sigma_{1}\rho_{1}H_{12}=\min\left(\sigma_{1}\rho_{1}H_{12},\ \frac{\sigma_{1}\rho_{1}|\theta_{1}-\theta_{2}|}{\Delta t},\ \frac{\sigma_{2}\rho_{2}|\theta_{1}-\theta_{2}|}{\Delta t}\right).
\end{equation}
This doesn't work \textendash{} it just makes the temperatures equilibrate
because there is only heat transfer when there is a difference. It
doesn't make them swap.

If mass is transferred based on its anomalous temperature, then it
makes sense that the mass transferred has a different mean temperature
to the mass left behind. However we know nothing about the sub-grid
scale variability and we are not solving equations involving the sub-grid
scale variability. Therefore the $H_{ij}$ term in equation \ref{eq:condThetaAdv}
will be used to transfer additional heat between partitions as this
term acts independently of the mass transfer. Considering again the
$\theta$ equation:
\begin{equation}
\frac{\partial\theta_{i}}{\partial t}+\mathbf{u}_{i}\cdot\nabla\theta_{i}=\sum_{j\ne i}\left(\frac{\sigma_{j}\rho_{j}}{\sigma_{i}\rho_{i}}S_{ji}(\theta_{j}-\theta_{i})-H_{ij}\right)
\end{equation}
and assuming that $H_{ij}$ can be positive and negative and that
$\sigma_{1}\rho_{1}H_{12}=-\sigma_{2}\rho_{2}H_{21}$ for energy conservation.
We want positive $\theta_{i}$ anomalies transferred to $j$ and negative
$\theta_{j}$ anomalies transferred to $i$. We assume that we are
using mass transfers:
\begin{eqnarray}
S_{12} & = & \begin{cases}
\max\left(-K_{\theta}\frac{\nabla^{2}\theta_{1}}{\theta_{1}},\ 1/\Delta t\right) & \ \text{when}\ \nabla^{2}\theta_{1}<0\\
0 & \ \text{otherwise}
\end{cases}\\
S_{21} & = & \begin{cases}
\max\left(K_{\theta}\frac{\nabla^{2}\theta_{2}}{\theta_{2}},\ 1/\Delta t\right) & \ \text{when}\ \nabla^{2}\theta_{2}>0\\
0 & \ \text{otherwise}
\end{cases}.
\end{eqnarray}
then we can use 
\begin{equation}
H_{ij}=\frac{\sigma_{i}\rho_{i}\theta_{i}S_{ij}-\sigma_{j}\rho_{j}\theta_{j}S_{ji}}{\sigma_{i}\rho_{i}}=\theta_{i}S_{ij}-\frac{\sigma_{j}\rho_{j}}{\sigma_{i}\rho_{i}}\theta_{j}S_{ji}
\end{equation}
This means that $\theta_{i}$ will always decrease and $\theta_{j}$
will always increase. This is what is wanted. This term should not
need to be treated implicitly but it will need limiting for stability.
For stability we need $\Delta tS_{ij}<1$ which implies that
\begin{eqnarray*}
\sigma_{i}\rho_{i}\theta_{i}|H_{ij}| & < & \sigma_{i}\rho_{i}\theta_{i}\\
\sigma_{j}\rho_{j}\theta_{j}|H_{ji}| & < & \sigma_{j}\rho_{j}\theta_{j}
\end{eqnarray*}
\clearpage{}

\section{Temperature of the Mass Transferred}

If the temperature of the transferred air is not the same as the temperature
of the fluid that is donating air, the flux form potential temperature
equation becomes:
\begin{equation}
\frac{\partial\sigma_{i}\rho_{i}\theta_{i}}{\partial t}+\nabla\cdot\left(\sigma_{i}\rho_{i}\mathbf{u}_{i}\theta_{i}\right)=\sum_{j\ne i}\left(\sigma_{j}\rho_{j}\theta_{j}^{T}S_{ji}-\sigma_{i}\rho_{i}\theta_{i}^{T}S_{ij}-\sigma_{i}\rho_{i}H_{ij}\right)
\end{equation}
where $\theta_{i}^{T}$ is the temperature of the air leaving fluid
$i$. Therefore the advective form is
\begin{equation}
\frac{\partial\theta_{i}}{\partial t}+\mathbf{u}_{i}\cdot\nabla\theta_{i}=\sum_{j\ne i}\left(\frac{\sigma_{j}\rho_{j}}{\sigma_{i}\rho_{i}}S_{ji}(\theta_{j}^{T}-\theta_{i})-(\theta_{i}^{T}-\theta_{i})S_{ij}-H_{ij}\right).
\end{equation}
If we want to effectively solve an advective form advection diffusion
equation for $\theta_{0}$:
\begin{equation}
\frac{\partial\theta_{0}}{\partial t}+\mathbf{u}_{0}\cdot\nabla\theta_{0}=K_{\theta}\nabla^{2}\theta_{0}
\end{equation}

$\alpha\beta\sum\frac{1}{2}$

assuming that $S_{01}>0$ and $S_{10}=0$ and $\nabla^{2}\theta_{0}<0$
then we can use:
\begin{equation}
S_{01}=-K_{\theta}\frac{\nabla^{2}\theta_{0}}{\theta_{0}^{t}}.
\end{equation}
In order to find $\theta_{0}^{t}$ we could set:
\begin{equation}
S_{01}=\begin{cases}
\frac{1}{\Delta t} & \ \text{if}\ \nabla^{2}\theta_{0}<0\\
0 & \ \text{otherwise}.
\end{cases}
\end{equation}
Equating these expressions for $S_{01}$ gives:
\begin{equation}
\theta_{t}^{0}=-K_{\theta}\Delta t\nabla^{2}\theta_{0}.
\end{equation}

Treating the transfer terms implicitly, ignoring $H_{ij}$ and writing
$\theta_{i}^{T}=\theta_{i}+\theta_{i}^{t}$ leads to
\begin{eqnarray}
\theta_{i}^{n+1} & = & \theta_{i}^{\prime}+\Delta t\sum_{j\ne i}\left(T_{ji}(\theta_{j}^{n+1}+\theta_{j}^{t}-\theta_{i}^{n+1})-\theta_{i}^{t}S_{ij}\right)\label{eq:condThetaAdv-1-1}
\end{eqnarray}
where $T_{ji}=\frac{\sigma_{j}\rho_{j}}{\sigma_{i}\rho_{i}}S_{ji}$.
If $i,j=1,2$ then this can be re-arranged as:
\begin{eqnarray}
\left(\begin{array}{cc}
1+\Delta t\ T_{21} & -\Delta t\ T_{21}\\
-\Delta t\ T_{12} & 1+\Delta t\ T_{12}
\end{array}\right)\left(\begin{array}{c}
\theta_{1}^{n+1}\\
\theta_{2}^{n+1}
\end{array}\right) & = & \left(\begin{array}{c}
\theta_{1}^{\prime}+\Delta t\left(T_{21}\theta_{2}^{t}-S_{12}\theta_{1}^{t}\right)\\
\theta_{2}^{\prime}+\Delta t\left(T_{12}\theta_{1}^{t}-S_{21}\theta_{2}^{t}\right)
\end{array}\right)
\end{eqnarray}
which can be solved implicitly for $\theta_{i}$ with construction
of only local two by two matrices, for example:
\begin{equation}
\left(\begin{array}{c}
\theta_{1}^{n+1}\\
\theta_{2}^{n+1}
\end{array}\right)=\frac{1}{1+\Delta t\ T_{21}+\Delta t\ T_{12}}\left(\begin{array}{cc}
1+\Delta t\ T_{12} & \Delta t\ T_{21}\\
\Delta t\ T_{12} & 1+\Delta t\ T_{21}
\end{array}\right)\left(\begin{array}{c}
\theta_{1}^{\prime}+\Delta t\left(T_{21}\theta_{2}^{t}-S_{12}\theta_{1}^{t}\right)\\
\theta_{2}^{\prime}+\Delta t\left(T_{12}\theta_{1}^{t}-S_{21}\theta_{2}^{t}\right)
\end{array}\right).
\end{equation}
Alternatively, to limit storage, $\theta_{1}^{n+1}$ and $\theta_{2}^{n+1}$
can be calculated as:
\begin{eqnarray*}
\theta_{1}^{n+1} & = & \left(\left(1+\Delta t\ T_{12}\right)\left(\theta_{1}^{\prime}+\Delta t\left(T_{21}\theta_{2}^{t}-S_{12}\theta_{1}^{t}\right)\right)+\Delta t\ T_{21}\left(\theta_{2}^{\prime}+\Delta t\left(T_{12}\theta_{1}^{t}-S_{21}\theta_{2}^{t}\right)\right)\right)\\
 &  & \bigg/\left(1+\Delta t\ T_{21}+\Delta t\ T_{12}\right)\\
\theta_{2}^{n+1} & = & \frac{\theta_{2}^{\prime}+\Delta t\left(T_{12}\theta_{1}^{t}-S_{21}\theta_{2}^{t}\right)+\Delta t\ T_{12}\theta_{1}^{n+1}}{1+\Delta tT_{12}}.
\end{eqnarray*}

\bibliographystyle{abbrvnat}
\bibliography{numerics}

\end{document}
